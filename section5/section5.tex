\documentclass[12pt,english]{article}
\usepackage{geometry}
\usepackage{float}
\usepackage{caption}
\geometry{verbose,tmargin=3cm,bmargin=3cm,lmargin=3cm,rmargin=3cm}
\usepackage{amsmath}
\usepackage{amssymb}
\usepackage{amsthm}
\usepackage{adjustbox}
\usepackage{hyperref}
\usepackage{graphicx}
\usepackage{setspace}
\usepackage{changepage}
\onehalfspacing
\usepackage{babel}
\newcommand{\expec}{\ensuremath{\mathbb E}}
\begin{document}
\begin{center}
{\Large{}Section 5: Moral hazard and adverse selection} \\
{\large{}Karlan and Zinman (2009)}
\par\end{center}{\Large \par}

\begin{center}
EEP 152
\par\end{center}

\begin{center}
September 28, 2016
\par\end{center}

\begin{itemize}
	\setlength\itemsep{-0.5em}
	\item Credit market failures (5 min)
	\item Moral hazard (20 min)
	\item Adverse selection (25 min)
\end{itemize}
A copy of a public version of the paper, along with the section notes, is available on the section Github at \href{github.com/johnloeser/eep152}{github.com/johnloeser/eep152} in the ``section5'' folder.

\section{Credit market failures}

We've seen in previous sections that credit market failures appear characteristic of developing countries. Microentrepreneurs in Sri Lanka and Ghana, small businesses in Nigeria, and larger firms in India all have marginal products of capital an order of magnitude higher than the interest rates on credit they have access to. In order to explain this, it must be the case that these firms are unable to take on as much credit as would be profit maximizing given market interest rates.\footnote{In a recent paper, Banerjee, Duflo, and Hornbeck argue that unmeasured costs, such as effort, may cause economists to overstate the effects of increased capital on profit for microentrepreneurs, but these are unlikely to be an issue for larger firms.}

This leads us to two questions that we'll focus on this week and in future weeks. First, what might explain why firms with high marginal products of capital are unable to take out loans at market (or above market) interest rates, resulting in profits for both banks and firms? Second, what solutions exist to correcting these credit market failures?

This week, we'll target the first question, in the hope that identifying the causes of the problem will help us to find solutions. We'll focus on two potential explanations of the credit market failure -- moral hazard, and adverse selection. It's worth noting that these two explanations are likely not sufficient to explain all credit market failures in developing countries -- imperfectly competitive and potentially corrupt banking industries, limited collateralizability of assets (potentially due to differences in the assets themselves or due to property rights), and limited credit histories, to give a few examples, may all be explanations in their own right or may worsen problems of moral hazard and adverse selection. Understanding the relative contributions of all of these potential problems is an active area of research and is far from settled. For today, we'll focus on moral hazard and adverse selection, and ask how we can test empirically their relevance.

\section{Moral hazard}

To remember the difference between moral hazard and adverse selection, it is useful to remember that moral hazard is an ex post information asymmetry problem, while adverse selection is an ex ante information asymmetry problem. For moral hazard, the bank knows that there is some probability that a borrower will be unable to repay the loan, if the borrower receives a negative shock, and the bank knows this probability (in contrast to adverse selection, where the bank is uncertain about this probability). However, the bank does not observe whether or not the borrower actually received a negative shock. As a result, the borrower can lie to the bank and default. Therefore, in order to make the borrower repay, the bank needs some way to punish the borrower for defaulting. For now, we'll assume that the bank will punish the borrower by refusing to lend to the borrower in the future. However, many other types of punishment are possible -- leaving a record on the borrower's credit history, seizing a collateralized asset, punishing the borrower's friends (we'll discuss this with microfinance), reducing the maximum size of future loans, \ldots A model with alternative modes of punishment yields similar results, so we'll focus on this case.

For the model, suppose a borrower and the bank play a repeated game. The borrower has a project they would like to undertake each period that costs $L$ and yields a return $R$. The bank lends to the borrower at an interest rate $i$. The bank wants the strongest possible incentive to prevent the borrower from defaulting -- therefore, the bank will stop lending to the borrower if the borrower defaults. The borrower is forward looking, and looks $N$ periods in advance. Therefore, we can write the profits of the borrower and the bank as below.

$$
\begin{array}{c|c|c|c}
& \text{No loan} & \text{Loan at } i & \text{Borrower default} \\
\hline
\text{Borrower} & 0 & (N + 1) (R - (1 + i)L) & R \\
\hline
\text{Bank} & 0 & iL \text{ (each period)} & -L
\end{array}
$$

First, we can see that the bank will not lend if the borrower will default. The borrower, facing a loan at interest rate $i$, will default if $R > (N + 1)(R - (1 + i)L)$. Simplifying this yields
$$ \text{default } \Leftrightarrow R < \frac{N+1}{N}(1+i)L $$
The borrower is more likely to default when they are impatient ($N$ is small), when the interest rate is high (since this reduces the profits associated with paying back the loan), and when $L$ is large relative to $R$ (when the project isn't very profitable, future borrowing to sustain it is not as attractive). The bank therefore will choose the maximum $i$ that does not push the borrower to default. In cases where this $i$ is negative (if $\frac{N+1}{N}L > R$), then the bank will not lend, and there will be a market failure.\footnote{As we saw in class, more flexible models where the borrower has a menu of projects they can choose from, or has a function $R = f(L)$, can generate credit constraints instead of a total market failure. The same is true of adverse selection. We won't discuss those models in section, but the intuition is similar.}

How can we test for the presence of moral hazard? The prediction of moral hazard is that the same borrower may choose to default or not default depending on bank behavior, while the prediction of adverse selection is that there are different borrowers with different default probabilities. While the two may exist simultaneously, we would like to be able to separately identify them.

To do this, Karlan and Zinman devise an experiment with a bank in South Africa focusing on low income workers. To control for selection, they send out flyers offering a constant interest rate to borrowers for a loan. However, after borrowers accepted or declined the loan, they then applied 2 interventions. First, some borrowers were subsequently surprised with a lower interest rate. Second, borrowers were also given an offer for a future loan, and the interest rate on this loan was varied. By waiting until after borrowers had accepted or declined the loan, they could guarantee that the pool of borrowers receiving the different interest rates were identical. The first treatment reduces the cost of paying back the current loan, while the second treatment increases the benefits of paying back the current loan, so both should reduce default if moral hazard is present. If random shocks affecting repayment really matter, then the first treatment should more strongly affect default rates than the second treatment, while if borrower effort to repay really matters, then the second treatment should more strongly affect default rates than the first treatment.

Karlan and Zinman find strong evidence of moral hazard from the second treatment -- borrowers receiving the dynamic incentive to repay (offered low interest rates on future loans conditional on repayment) are significantly more likely to stay up-to-date on their repayments than borrowers without the dynamic incentives. In contrast, reducing interest rates on the current loan has a much smaller effect on repayment (again, conditional on selection).

\section{Adverse selection}

In contrast, adverse selection is an ex ante information asymmetry problem -- borrowers have fixed probabilities of defaulting, but banks do not know the default probabilities of borrowers. This creates a problem because the interest rate is effectively lower for borrowers with a high default probability (since borrowers pay no interest when they default), and as a result these borrowers are willing to take out loans at higher interest rates. Thus, the bank is not profitable at low interest rates due to default, but raising interest rates drives out the low default probability borrowers, increasing default rates and (potentially) decreasing profits. As a result, a market failure might result where the bank does not lend, or alternatively where the bank only lends to risky borrowers.

To model this, suppose there are two types of borrowers -- a safe borrower with probability 0 of default, and a risky borrower with probability $p$ of default, with a proportion $\frac{1}{2}$ of each. Both borrowers would like to undertake a project with cost $L$, but the return for the safe borrower is $R$, while the return for the risky borrower is $R' > R$. To maximize profits, the bank will consider two interest rates -- $i = \frac{R - L}{L}$ (the highest interest rate at which the safe borrower is willing to borrow) and $i' = \frac{R' - L}{L}$ (the highest interest rate at which the risky borrower is willing to borrow). We can then calculate the profits the bank earns from each borrower at each interest rate.

$$
\begin{array}{c|c|c}
& \text{Safe borrower} & \text{Risky borrower} \\
\hline
i & R - L & pR - L \\
\hline
i' & 0 & pR' - L
\end{array}
$$

The bank earns 0 profits when it doesn't lend (infinite interest rate), $\frac{1}{2}(1 + p)R - L$ when it lends at $i$, and $\frac{1}{2}(pR' - L)$ when it lends at $i'$. Most notably, a high probability of default among risky borrowers reduces the probability that the bank can profitably lend to safe borrowers.

When testing for moral hazard, Karlan and Zinman wanted to control for selection, while varying the ex post incentives to default. In constrast, to test for adverse selection, Karlan and Zinman want to allow for selection, but hold the ex post incentives to default constant. To do this, they compare groups of borrowers that are differentially selected, but eventually offered the same interest rate -- borrowers whose flyers offered a high interest rate, but are eventually surprised with a low interest rate, and borrowers whose flyers offered a low interest rate. Although they lack statistical power for this test, they find evidence that there is some adverse selection (default rates are higher for higher offered interest rates), but the magnitude of the effect of adverse selection on default appears small compared to the magnitude of the effect of moral hazard.

\end{document}